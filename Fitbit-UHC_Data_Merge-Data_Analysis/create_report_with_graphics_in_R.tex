\documentclass[]{article}
\usepackage{lmodern}
\usepackage{amssymb,amsmath}
\usepackage{ifxetex,ifluatex}
\usepackage{fixltx2e} % provides \textsubscript
\ifnum 0\ifxetex 1\fi\ifluatex 1\fi=0 % if pdftex
  \usepackage[T1]{fontenc}
  \usepackage[utf8]{inputenc}
\else % if luatex or xelatex
  \ifxetex
    \usepackage{mathspec}
  \else
    \usepackage{fontspec}
  \fi
  \defaultfontfeatures{Ligatures=TeX,Scale=MatchLowercase}
\fi
% use upquote if available, for straight quotes in verbatim environments
\IfFileExists{upquote.sty}{\usepackage{upquote}}{}
% use microtype if available
\IfFileExists{microtype.sty}{%
\usepackage{microtype}
\UseMicrotypeSet[protrusion]{basicmath} % disable protrusion for tt fonts
}{}
\usepackage[margin=1in]{geometry}
\usepackage{hyperref}
\hypersetup{unicode=true,
            pdftitle={Fitbit Members in United: Exploration},
            pdfauthor={Seth Grossinger},
            pdfborder={0 0 0},
            breaklinks=true}
\urlstyle{same}  % don't use monospace font for urls
\usepackage{longtable,booktabs}
\usepackage{graphicx,grffile}
\makeatletter
\def\maxwidth{\ifdim\Gin@nat@width>\linewidth\linewidth\else\Gin@nat@width\fi}
\def\maxheight{\ifdim\Gin@nat@height>\textheight\textheight\else\Gin@nat@height\fi}
\makeatother
% Scale images if necessary, so that they will not overflow the page
% margins by default, and it is still possible to overwrite the defaults
% using explicit options in \includegraphics[width, height, ...]{}
\setkeys{Gin}{width=\maxwidth,height=\maxheight,keepaspectratio}
\IfFileExists{parskip.sty}{%
\usepackage{parskip}
}{% else
\setlength{\parindent}{0pt}
\setlength{\parskip}{6pt plus 2pt minus 1pt}
}
\setlength{\emergencystretch}{3em}  % prevent overfull lines
\providecommand{\tightlist}{%
  \setlength{\itemsep}{0pt}\setlength{\parskip}{0pt}}
\setcounter{secnumdepth}{0}
% Redefines (sub)paragraphs to behave more like sections
\ifx\paragraph\undefined\else
\let\oldparagraph\paragraph
\renewcommand{\paragraph}[1]{\oldparagraph{#1}\mbox{}}
\fi
\ifx\subparagraph\undefined\else
\let\oldsubparagraph\subparagraph
\renewcommand{\subparagraph}[1]{\oldsubparagraph{#1}\mbox{}}
\fi

%%% Use protect on footnotes to avoid problems with footnotes in titles
\let\rmarkdownfootnote\footnote%
\def\footnote{\protect\rmarkdownfootnote}

%%% Change title format to be more compact
\usepackage{titling}

% Create subtitle command for use in maketitle
\newcommand{\subtitle}[1]{
  \posttitle{
    \begin{center}\large#1\end{center}
    }
}

\setlength{\droptitle}{-2em}

  \title{Fitbit Members in United: Exploration}
    \pretitle{\vspace{\droptitle}\centering\huge}
  \posttitle{\par}
    \author{Seth Grossinger}
    \preauthor{\centering\large\emph}
  \postauthor{\par}
      \predate{\centering\large\emph}
  \postdate{\par}
    \date{August 3, 2018}


\begin{document}
\maketitle

\subsection{Preliminary work}\label{preliminary-work}

Before we look at the characteristics of members with Fitbits I would
like to better understand which members to potentially include.
Initially we're interested in broad comparisons, and if we find
something interesting we can further drill down. I'm looking at a single
year of claims for adults members and looking at

\begin{itemize}
\tightlist
\item
  Age/Gender
\item
  Utilzation

  \begin{itemize}
  \tightlist
  \item
    average count of Inpatient hospital admissions
  \item
    average count of ER visits that did not result in an admission
  \item
    PMPM allowable amounts, totals and Medical/Rx breakdowns
  \item
    Charlson score
  \item
    Charlson indicators.
  \end{itemize}
\end{itemize}

I am \emph{not} currently including RAF/HCCs. We can drill in further in
the future, but the Charlson scores I'm using are much easier (at least
computationally) to compute. Either serves as an imperfect proxy for
health status, so for now we'll save RAF/HCCs for future exploration. In
addition, the sample sizes we're working with are large, so I can afford
to paint with a broad brush and use simple rules. For example, I'm using
+ Adults = born before 1999 + all members continuously enrolled in the
same plan during 2017 + Charlson scores and indicators built requiring
only a single diagnosis during 2017, with no filters based on visit
type, place of service, provider or the like.

As an example of how to interpret these scores, `Diabetes' for a member
wouldn't indicate that the member meets clinical criteria for diabetes,
just that she had at least one diagnosis in 2017 related to diabetes. If
the `Diabetes' rate in a group of members is 6.8\%, that just means it's
likely that more people in the group have diabetes than in a different
group with a 5.9\% rate.

\paragraph{Members I'll include}\label{members-ill-include}

Different categories of insurance have different utilization patterns.
Also, again to keep things simple I'm using MiniHPDM as our data source.
There may be groups or categories for which our data is better than
others, so I want to make sure I pick members for whom I have at least
some confidence in the completeness (and comparability) of our data.

To start, I grouped members by their group's company (United
Healthcare/Uniprise) and financing arrangment (ASO/FI/\ldots{}) as
indicated in our membership files.

\begin{longtable}[]{@{}lllrlllrrlllrlll@{}}
\toprule
Company & Fund & MbrCnt & Age & Female & HasEmail & HasFitbit & IP.Admit
& ER.Visit & PMPM & Med & Rx & Charlson & Diabetes & HTN &
Depression\tabularnewline
\midrule
\endhead
UNIPRISE & ASO & 6,577,414 & 46.0 & 50.5\% & 53.7\% & 5.6\% & 0.05 &
0.18 & \$430 & \$402 & \$28 & 0.64 & 6.9\% & 18.2\% &
5.9\%\tabularnewline
UHC & ASO & 3,014,557 & 44.1 & 50.1\% & 49.9\% & 5.2\% & 0.05 & 0.17 &
\$472 & \$405 & \$68 & 0.64 & 7.2\% & 18.2\% & 5.9\%\tabularnewline
UHC & FI & 2,494,049 & 43.3 & 48.4\% & 56.7\% & 5.6\% & 0.05 & 0.18 &
\$481 & \$379 & \$101 & 0.60 & 6.6\% & 17.7\% & 6.0\%\tabularnewline
NULL & NULL & 1,663,628 & 67.4 & 55.6\% & 20.5\% & 1.4\% & 0.00 & 0.02 &
\$43 & \$33 & \$9 & 0.05 & 0.6\% & 1.7\% & 0.6\%\tabularnewline
UNIPRISE & FI & 222,334 & 55.1 & 47.3\% & 45.5\% & 3.9\% & 0.09 & 0.25 &
\$459 & \$382 & \$77 & 1.19 & 13.2\% & 32.6\% & 6.6\%\tabularnewline
NULL & NULL & 170,237 & 45.6 & 52.9\% & 9.1\% & 0.9\% & 0.00 & 0.00 &
\$2 & \$2 & \$0 & 0.00 & 0.0\% & 0.1\% & 0.0\%\tabularnewline
UHC & PI & 156,429 & 42.5 & 49.1\% & 61.1\% & 6.2\% & 0.05 & 0.15 &
\$545 & \$434 & \$111 & 0.54 & 5.2\% & 14.4\% & 6.4\%\tabularnewline
MAMSI-UHC & FI & 64,701 & 48.8 & 50.6\% & 51.0\% & 4.8\% & 0.06 & 0.24 &
\$517 & \$355 & \$161 & 1.00 & 12.7\% & 29.4\% & 6.0\%\tabularnewline
UNIPRISE & PI & 3,212 & 63.2 & 49.4\% & 42.1\% & 4.2\% & 0.12 & 0.29 &
\$396 & \$396 & \$0 & 1.49 & 14.6\% & 39.4\% & 8.9\%\tabularnewline
\bottomrule
\end{longtable}

For Fully Insured we typically restrict to UHC/FI. It seems like both
Uniprise and UHC are potentially usable for both ASO and FI. I've seen
misleading figures for UHC-ASO in the past, but to be sure I took a look
at the 10 largest customer segments within each grouping.

\paragraph{UHC FI}\label{uhc-fi}

There are a few groups that look like they have abnormally low
utilization, but no group accounts for a large share of overall
membership of 2.5MM and in aggregate the numbers look reasonable. This
is the segment we typically focus in our commercial analyses.

\begin{longtable}[]{@{}lrlrrlr@{}}
\toprule
MbrCnt & Age & Female & IP.Admit & ER.Visit & PMPM &
Charlson\tabularnewline
\midrule
\endhead
2,494,049 & 43.3 & 48.4\% & 0.05 & 0.18 & \$481 & 0.6\tabularnewline
\bottomrule
\end{longtable}

\begin{longtable}[]{@{}rllrlrrlr@{}}
\toprule
Rnk & CUST\_SEG\_NM & MbrCnt & Age & Female & IP.Admit & ER.Visit & PMPM
& Charlson\tabularnewline
\midrule
\endhead
1 & NORTHSIDE INDEPENDENT SCHOOL DISTRICT & 13,375 & 42.0 & 64.4\% &
0.05 & 0.17 & \$421 & 0.66\tabularnewline
2 & SOUTHWEST CARPENTERS HEALTH AND WELFARE TRUST (MEDICAL) & 12,748 &
40.0 & 43.8\% & 0.01 & 0.07 & \$117 & 0.15\tabularnewline
3 & STATE OF WISCONSIN & 9,453 & 46.1 & 52.6\% & 0.06 & 0.21 & \$518 &
0.81\tabularnewline
4 & ARIZONA STATE RETIREMENT SYSTEM & 8,661 & 58.2 & 59.7\% & 0.05 &
0.16 & \$697 & 1.11\tabularnewline
5 & DISTRICT OF COLUMBIA GOVERNMENT & 7,805 & 41.0 & 56.3\% & 0.06 &
0.33 & \$535 & 0.77\tabularnewline
6 & GENERAL ATOMICS & 7,063 & 44.6 & 45.7\% & 0.02 & 0.05 & \$177 &
0.23\tabularnewline
7 & AT\&T & 6,796 & 39.0 & 45.6\% & 0.05 & 0.28 & \$501 &
0.64\tabularnewline
8 & VIRGINIAS HEALTH INSURANCE MARKETPLACE & 6,534 & 43.8 & 55.8\% &
0.05 & 0.22 & \$505 & 0.75\tabularnewline
9 & CITY AND COUNTY OF DENVER & 5,151 & 44.1 & 51.5\% & 0.05 & 0.19 &
\$642 & 0.67\tabularnewline
10 & FEDERAL EMPLOYEES HEALTH BENEFITS PROGRAM LR & 4,923 & 43.8 &
53.1\% & 0.04 & 0.18 & \$409 & 0.70\tabularnewline
\bottomrule
\end{longtable}

\paragraph{UHC ASO}\label{uhc-aso}

Contrary to my expectation, overall utilization for this group looks
reasonable. However, several of the largest groups (Royal Caribbean,
BD-BDIL, Lloyd's of London) are foreign insurance companies (or cruise
lines) that cover their members when travelling in the US through a
reciprocal agreement. These groups each account for over 1\% of overall
members and have clearly different utilization. I'll exclude them from
this analysis, though it'd be good to try to understand this group and
whether there's a way to make a finer-grain decision about who to
include in the future.

\begin{longtable}[]{@{}lrlrrlr@{}}
\toprule
MbrCnt & Age & Female & IP.Admit & ER.Visit & PMPM &
Charlson\tabularnewline
\midrule
\endhead
3,014,557 & 44.1 & 50.1\% & 0.05 & 0.17 & \$472 & 0.64\tabularnewline
\bottomrule
\end{longtable}

\begin{longtable}[]{@{}rllrlrrlr@{}}
\toprule
Rnk & CUST\_SEG\_NM & MbrCnt & Age & Female & IP.Admit & ER.Visit & PMPM
& Charlson\tabularnewline
\midrule
\endhead
1 & ROYAL CARIBBEAN CRUISES LTD. & 66,953 & 36.1 & 20.9\% & 0.00 & 0.01
& \$23 & 0.01\tabularnewline
2 & BD-BDIL & 60,617 & 45.5 & 53.8\% & 0.01 & 0.01 & \$70 &
0.04\tabularnewline
3 & STATE OF ARIZONA & 53,506 & 47.2 & 54.3\% & 0.06 & 0.22 & \$477 &
0.97\tabularnewline
4 & SEVEN CORNERS - LLOYDS OF LONDON & 48,250 & 38.3 & 59.6\% & 0.00 &
0.00 & \$2 & 0.00\tabularnewline
5 & NORTHWELL HEALTH & 41,497 & 43.5 & 59.0\% & 0.08 & 0.17 & \$658 &
0.78\tabularnewline
6 & THE SCHOOL DISTRICT OF PALM BEACH COUNTY & 26,586 & 45.5 & 62.3\% &
0.05 & 0.21 & \$551 & 0.85\tabularnewline
7 & UNIVERSITY OF MISSOURI & 26,425 & 44.5 & 54.9\% & 0.06 & 0.18 &
\$409 & 0.68\tabularnewline
8 & STATE OF RHODE ISLAND & 26,345 & 48.1 & 52.4\% & 0.07 & 0.22 & \$510
& 1.02\tabularnewline
9 & CITY OF AUSTIN & 23,878 & 47.2 & 44.4\% & 0.07 & 0.24 & \$648 &
0.94\tabularnewline
10 & PORT AUTHORITY OF NY \& NJ & 23,660 & 55.9 & 47.9\% & 0.09 & 0.24 &
\$544 & 1.38\tabularnewline
\bottomrule
\end{longtable}

\paragraph{Uniprise ASO}\label{uniprise-aso}

These comprise the large national companies you'd think of as likely ASO
clients. Interestingly, while the overall utilization for this group
appears reasonable the largest group by far, AT\&T Care Plus, accounts
for a full 8\% of the membership and is a clear outlier. This
\href{https://careplus.att.com/}{appears} to be a supplemental policy
and should probably be excluded in the future. (And the next largest
group also appears to be an outlier.)

\begin{longtable}[]{@{}lrlrrlr@{}}
\toprule
MbrCnt & Age & Female & IP.Admit & ER.Visit & PMPM &
Charlson\tabularnewline
\midrule
\endhead
6,577,414 & 46 & 50.5\% & 0.05 & 0.18 & \$430 & 0.64\tabularnewline
\bottomrule
\end{longtable}

\begin{longtable}[]{@{}rllrlrrlr@{}}
\toprule
Rnk & CUST\_SEG\_NM & MbrCnt & Age & Female & IP.Admit & ER.Visit & PMPM
& Charlson\tabularnewline
\midrule
\endhead
1 & AT\&T INC.-CARE PLUS & 532,826 & 59.2 & 53.7\% & 0.01 & 0.05 & \$134
& 0.21\tabularnewline
2 & RAILROAD EMPLOYEES NATIONAL HEALTH \& WELFARE PLAN & 196,640 & 41.0
& 45.1\% & 0.03 & 0.14 & \$279 & 0.36\tabularnewline
3 & WAL-MART STORES INC. ASSOCIATES HEALTH AND WELFARE PLAN & 139,102 &
43.4 & 54.7\% & 0.06 & 0.30 & \$404 & 0.72\tabularnewline
4 & JPMORGAN CHASE \& CO. & 132,235 & 42.3 & 53.8\% & 0.05 & 0.16 &
\$467 & 0.62\tabularnewline
5 & DELTA AIR LINES INC & 119,226 & 46.2 & 50.8\% & 0.05 & 0.16 & \$605
& 0.70\tabularnewline
6 & WELLS FARGO & 116,456 & 41.9 & 54.0\% & 0.05 & 0.17 & \$446 &
0.59\tabularnewline
7 & SOUTHWEST AIRLINES COMPANY & 84,090 & 43.4 & 50.0\% & 0.04 & 0.18 &
\$483 & 0.62\tabularnewline
8 & GENERAL ELECTRIC & 82,701 & 45.5 & 45.7\% & 0.05 & 0.15 & \$420 &
0.61\tabularnewline
9 & RAYTHEON COMPANY & 80,854 & 45.8 & 47.4\% & 0.05 & 0.14 & \$484 &
0.69\tabularnewline
10 & AMERICAN AIRLINES & 80,011 & 59.1 & 48.4\% & 0.06 & 0.17 & \$301 &
1.01\tabularnewline
\bottomrule
\end{longtable}

\paragraph{Uniprise FI}\label{uniprise-fi}

We've typically excluded these companies in the past from our commercial
analyses. I'm not sure what these represent. The members are relatively
old, which could account for the high utilization/Charlson scores. This
is largely driven by the largest customer segment, with over 1/3 of the
total members. There are relatively few members overall, especially in
relation to the companies they represent. For example, there are 7,000
members from Starbucks and 16,000 from HP Enterprise, small percentages
of their overall workforces. We'll exclude them for now, but this is
another group of customers it would be good to better understand in the
future.

\begin{longtable}[]{@{}lrlrrlr@{}}
\toprule
MbrCnt & Age & Female & IP.Admit & ER.Visit & PMPM &
Charlson\tabularnewline
\midrule
\endhead
222,334 & 55.1 & 47.3\% & 0.09 & 0.25 & \$459 & 1.19\tabularnewline
\bottomrule
\end{longtable}

\begin{longtable}[]{@{}rllrlrrlr@{}}
\toprule
Rnk & CUST\_SEG\_NM & MbrCnt & Age & Female & IP.Admit & ER.Visit & PMPM
& Charlson\tabularnewline
\midrule
\endhead
1 & HEALTH AND WELFARE COMMITTEE COOPERATING RAILWAY LABOR ORGANIZATIONS
& 83,717 & 74.5 & 42.6\% & 0.16 & 0.31 & \$280 & 1.99\tabularnewline
2 & HEWLETT PACKARD ENTERPRISE COMPANY & 16,211 & 45.2 & 49.0\% & 0.04 &
0.14 & \$554 & 0.69\tabularnewline
3 & ARAMARK CORPORATION & 8,662 & 46.6 & 54.4\% & 0.07 & 0.29 & \$667 &
0.92\tabularnewline
4 & STARBUCKS CORPORATION & 7,025 & 35.5 & 58.3\% & 0.04 & 0.15 & \$355
& 0.31\tabularnewline
5 & FIRST DATA CORPORATION & 6,846 & 43.2 & 52.0\% & 0.05 & 0.17 & \$530
& 0.69\tabularnewline
6 & RICOH AMERICAS CORPORATION & 5,688 & 46.4 & 48.8\% & 0.06 & 0.20 &
\$697 & 0.88\tabularnewline
7 & HILTON DOMESTIC OPERATING COMPANY INC. & 5,547 & 45.3 & 54.6\% &
0.07 & 0.22 & \$650 & 0.87\tabularnewline
8 & AON SERVICE CORPORATION & 5,053 & 42.1 & 59.1\% & 0.06 & 0.20 &
\$616 & 0.64\tabularnewline
9 & DARDEN RESTAURANTS INC. & 4,787 & 40.3 & 47.8\% & 0.05 & 0.26 &
\$478 & 0.54\tabularnewline
10 & WOOD GROUP U.S. HOLDINGS INC. & 4,742 & 42.6 & 43.5\% & 0.05 & 0.17
& \$580 & 0.65\tabularnewline
\bottomrule
\end{longtable}

\subsection{Initial Fitbit member
exploration}\label{initial-fitbit-member-exploration}

We matched email addresses of Unitedhealth commercial members with
Fitbit members.

As an initial question, whose email addresses do we have? I know that
claims vary by age, by gender, and by funding arrangment (fully-insured
vs self-funded/ASO). Since we're potentially matching only members whose
email addresses we have, let's make sure they're similar to the overall
membership before proceed to a comparison of Fitbit members.

\includegraphics{create_report_with_graphics_in_R_files/figure-latex/unnamed-chunk-8-1.pdf}

Overall, it seems like we're slightly more likely to have email
addresses for women in FI and for men in ASO, but I'm not sure what that
tells us. However, when we look at utilization a problem becomes
evident.

\includegraphics{create_report_with_graphics_in_R_files/figure-latex/unnamed-chunk-9-1.pdf}
\includegraphics{create_report_with_graphics_in_R_files/figure-latex/unnamed-chunk-9-2.pdf}

The pattern is the same for FI and ASO: members whose email addresses we
know are significantly (in a practical if not statistical sense--I
haven't performed any statistical tests) more likely to have had a
diagnosis related to virtually every condition. We see this in
utilization statistics as well.

\begin{longtable}[]{@{}lllrlrrlllrlll@{}}
\toprule
Insurance & HaveEmail & MbrCnt & Age & Female & IP.Admit & ER.Visit &
PMPM & Med & Rx & Charlson & Diabetes & HTN & Depression\tabularnewline
\midrule
\endhead
UHC-FI & No & 1,080,348 & 41.2 & 45.9\% & 0.040 & 0.179 & \$338 & \$288
& \$50 & 0.46 & 5.1\% & 13.6\% & 4.6\%\tabularnewline
UHC-FI & Yes & 1,413,701 & 45.0 & 50.3\% & 0.052 & 0.173 & \$589 & \$449
& \$140 & 0.71 & 7.7\% & 20.8\% & 7.1\%\tabularnewline
UHC-FI & Total & 2,494,049 & 43.3 & 48.4\% & 0.047 & 0.176 & \$481 &
\$379 & \$101 & 0.60 & 6.6\% & 17.7\% & 6\%\tabularnewline
\bottomrule
\end{longtable}

\begin{longtable}[]{@{}lllrlrrlllrlll@{}}
\toprule
Insurance & HaveEmail & MbrCnt & Age & Female & IP.Admit & ER.Visit &
PMPM & Med & Rx & Charlson & Diabetes & HTN & Depression\tabularnewline
\midrule
\endhead
Uni-ASO & No & 3,045,962 & 44.4 & 53.7\% & 0.049 & 0.183 & \$336 & \$325
& \$12 & 0.56 & 5.8\% & 15.3\% & 5.2\%\tabularnewline
Uni-ASO & Yes & 3,531,452 & 47.3 & 47.8\% & 0.055 & 0.169 & \$510 &
\$469 & \$41 & 0.72 & 7.8\% & 20.7\% & 6.4\%\tabularnewline
Uni-ASO & Total & 6,577,414 & 46.0 & 50.5\% & 0.052 & 0.175 & \$430 &
\$402 & \$28 & 0.64 & 6.9\% & 18.2\% & 5.9\%\tabularnewline
\bottomrule
\end{longtable}

I don't know the sources of our email addresses, but it makes intuitive
sense. If you're healthy and have no claims or need to find a provider
we're probably less likely to know your email address. If you have many
claims to manage and need to find many different providers there's
probably a good chance we have your email address.

Given all of that, though, it's clear that we shouldn't compare members
who have Fitbits with all other members. Since we would only potentially
know if a member has a Fitbit if we know the member's email address, we
should compare members who have Fitbits only to other members whose
email addresses we have. So let's go ahead and do that.

\subsection{Explore members with (and without)
Fitbits}\label{explore-members-with-and-without-fitbits}

\includegraphics{create_report_with_graphics_in_R_files/figure-latex/unnamed-chunk-11-1.pdf}

\begin{longtable}[]{@{}lllrlrrlllrlll@{}}
\toprule
Insurance & HaveFitbit & MbrCnt & Age & Female & IP.Admit & ER.Visit &
PMPM & Med & Rx & Charlson & Diabetes & HTN & Depression\tabularnewline
\midrule
\endhead
UHC-FI & No & 1,275,001 & 45.1 & 49\% & 0.052 & 0.177 & \$589 & \$449 &
\$140 & 0.71 & 7.9\% & 21\% & 7\%\tabularnewline
UHC-FI & Yes & 138,700 & 44.6 & 62.7\% & 0.049 & 0.138 & \$595 & \$452 &
\$143 & 0.66 & 6.4\% & 19.2\% & 7.8\%\tabularnewline
UHC-FI & Total & 1,413,701 & 45.0 & 50.3\% & 0.052 & 0.173 & \$589 &
\$449 & \$140 & 0.71 & 7.7\% & 20.8\% & 7.1\%\tabularnewline
\bottomrule
\end{longtable}

\begin{longtable}[]{@{}lllrlrrlllrlll@{}}
\toprule
Insurance & HaveFitbit & MbrCnt & Age & Female & IP.Admit & ER.Visit &
PMPM & Med & Rx & Charlson & Diabetes & HTN & Depression\tabularnewline
\midrule
\endhead
Uni-ASO & No & 3,165,013 & 47.4 & 46.7\% & 0.055 & 0.172 & \$509 & \$468
& \$41 & 0.72 & 7.9\% & 20.8\% & 6.4\%\tabularnewline
Uni-ASO & Yes & 366,439 & 46.8 & 57.5\% & 0.049 & 0.137 & \$521 & \$481
& \$41 & 0.68 & 6.7\% & 19.9\% & 6.8\%\tabularnewline
Uni-ASO & Total & 3,531,452 & 47.3 & 47.8\% & 0.055 & 0.169 & \$510 &
\$469 & \$41 & 0.72 & 7.8\% & 20.7\% & 6.4\%\tabularnewline
\bottomrule
\end{longtable}


\end{document}
